\documentclass{beamer}
\mode<presentation>
\usepackage{amsmath}
\usepackage{amssymb}
%\usepackage{advdate}
\usepackage{graphicx}
\graphicspath{{../Figs/}}
\usepackage{adjustbox}
\usepackage{subcaption}
\usepackage{enumitem}
\usepackage{multicol}
\usepackage{mathtools}
\usepackage{listings}
\usepackage{url}
\def\UrlBreaks{\do\/\do-}
\usetheme{Boadilla}
\usecolortheme{lily}
\setbeamertemplate{footline}
{
  \leavevmode%
  \hbox{%
  \begin{beamercolorbox}[wd=\paperwidth,ht=2.25ex,dp=1ex,right]{author in head/foot}%
    \insertframenumber{} / \inserttotalframenumber\hspace*{2ex} 
  \end{beamercolorbox}}%
  \vskip0pt%
}
\setbeamertemplate{navigation symbols}{}
\let\solution\relax
\usepackage{gvv}
\lstset{
%language=C,
frame=single, 
breaklines=true,
columns=fullflexible
}

\numberwithin{equation}{section}
\begin{document}

\title{12.195}
\author{AI25BTECH11002 - Ayush Sunil Labhade}
{\let\newpage\relax\maketitle}

\textbf{Question :} Let \(T:R^4 to R^4\) be the linear map satisfying
\[
T(e_1)=e_2,\qquad T(e_2)=e_3,\qquad T(e_3)=0,\qquad T(e_4)=e_3,
\]
where \(\{e_1,e_2,e_3,e_4\}\) is the standard basis of \(\R^4\). Then determine which of the following statements are true:
\begin{enumerate}[label=(\alph*)]
  \item \(T\) is idempotent.
  \item \(T\) is invertible.
  \item \(\operatorname{rank} T = 3\).
  \item \(T\) is nilpotent.
\end{enumerate}

\bigskip

\textbf{Solution:} \newline
The matrix of \(T\) has its \(j\)th column equal to the coordinates of \(T(e_j)\). Hence
	$[T]_{\{e_i\}}$
\begin{align}
\myvec{
0 & 0 & 0 & 0 \\
1 & 0 & 0 & 0 \\
0 & 1 & 0 & 1 \\
0 & 0 & 0 & 0
}
\end{align}

\medskip
Compute \(T^2\):
\begin{align}
\myvec{
0 & 0 & 0 & 0 \\
0 & 0 & 0 & 0 \\
1 & 0 & 0 & 0 \\
0 & 0 & 0 & 0
}
\end{align}
So \(T^2\) is not the zero. \newline
Now compute \(T^3\):
\begin{align}
\myvec{
0 & 0 & 0 & 0 \\
0 & 0 & 0 & 0 \\
0 & 0 & 0 & 0 \\
0 & 0 & 0 & 0
}
\end{align}
$\since T^3$ is a null matrix \newline
$\therefore$ T is a nilpotent matrix.\newline
\medskip


\end{document}

