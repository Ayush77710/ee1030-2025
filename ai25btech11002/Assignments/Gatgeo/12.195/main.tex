\let\negmedspace\undefined
\let\negthickspace\undefined
\documentclass[journal,12pt,onecolumn]{IEEEtran}
\usepackage{cite}
\usepackage{amsmath,amssymb,amsfonts,amsthm}
\usepackage{algorithmic}
\usepackage{graphicx}
\graphicspath{{Figs/}}
\usepackage{textcomp}
\usepackage{xcolor}
\usepackage{txfonts}
\usepackage{listings}
\usepackage{enumitem}
\usepackage{mathtools}
\usepackage{gensymb}
\usepackage{comment}
\usepackage{caption}
\usepackage[breaklinks=true]{hyperref}
\usepackage{tkz-euclide} 
\usepackage{listings}
\usepackage{gvv}                                        
\usepackage[latin1]{inputenc}     
\usepackage{xparse}
\usepackage{color}                                            
\usepackage{array}                                            
\usepackage{longtable}                                       
\usepackage{calc}                                             
\usepackage{multirow}
\usepackage{multicol}
\usepackage{hhline}                                           
\usepackage{ifthen}                                           
\usepackage{lscape}
\usepackage{tabularx}
\usepackage{array}
\usepackage{float}

\newcommand{\R}{\mathbb{R}}
% \myvec and other macros are assumed to be defined in your original main.tex / gvv package.
% If not, you can define a simple fallback:
\providecommand{\myvec}[1]{\begin{pmatrix}#1\end{pmatrix}}

\begin{document}

\title{12.195}
\author{AI25BTECH11002 - Ayush Sunil Labhade}
{\let\newpage\relax\maketitle}

\textbf{Question :} Let \(T:\R^4\to\R^4\) be the linear map satisfying
\[
T(e_1)=e_2,\qquad T(e_2)=e_3,\qquad T(e_3)=0,\qquad T(e_4)=e_3,
\]
where \(\{e_1,e_2,e_3,e_4\}\) is the standard basis of \(\R^4\). Then determine which of the following statements are true:
\begin{enumerate}[label=(\alph*)]
  \item \(T\) is idempotent.
  \item \(T\) is invertible.
  \item \(\operatorname{rank} T = 3\).
  \item \(T\) is nilpotent.
\end{enumerate}

\bigskip

\textbf{Solution:} \newline
The matrix of \(T\) has its \(j\)th column equal to the coordinates of \(T(e_j)\). Hence
	$[T]_{\{e_i\}}$
\begin{align}
\myvec{
0 & 0 & 0 & 0 \\
1 & 0 & 0 & 0 \\
0 & 1 & 0 & 1 \\
0 & 0 & 0 & 0
}
\end{align}

\medskip
Compute \(T^2\):
\begin{align}
\myvec{
0 & 0 & 0 & 0 \\
0 & 0 & 0 & 0 \\
1 & 0 & 0 & 0 \\
0 & 0 & 0 & 0
}
\end{align}
So \(T^2\) is not the zero. \newline
Now compute \(T^3\):
\begin{align}
\myvec{
0 & 0 & 0 & 0 \\
0 & 0 & 0 & 0 \\
0 & 0 & 0 & 0 \\
0 & 0 & 0 & 0
}
\end{align}
$\since T^3$ is a null matrix \newline
$\therefore$ T is a nilpotent matrix.\newline
\medskip


\end{document}

