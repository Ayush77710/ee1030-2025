\documentclass{beamer}
\mode<presentation>
\usepackage{amsmath}
\usepackage{amssymb}
%\usepackage{advdate}
\usepackage{graphicx}
\graphicspath{{../Figs/}}
\usepackage{adjustbox}
\usepackage{subcaption}
\usepackage{enumitem}
\usepackage{multicol}
\usepackage{mathtools}
\usepackage{listings}
\usepackage{url}
\def\UrlBreaks{\do\/\do-}
\usetheme{Boadilla}
\usecolortheme{lily}
\setbeamertemplate{footline}
{
  \leavevmode%
  \hbox{%
  \begin{beamercolorbox}[wd=\paperwidth,ht=2.25ex,dp=1ex,right]{author in head/foot}%
    \insertframenumber{} / \inserttotalframenumber\hspace*{2ex} 
  \end{beamercolorbox}}%
  \vskip0pt%
}
\setbeamertemplate{navigation symbols}{}
\let\solution\relax
\usepackage{gvv}
\lstset{
%language=C,
frame=single, 
breaklines=true,
columns=fullflexible
}

\numberwithin{equation}{section}


\begin{document}

\title{12.424}
\author{AI25BTECH11002 - Ayush Sunil Labhade}
{\let\newpage\relax\maketitle}

\textbf{Question:}\\
Let \(T:\mathbb{R}^4\to\mathbb{R}^4\) be the linear map defined by
\[
T(x,y,z,w)=(\,x+z,\;2x+y+3z,\;2y+2z,\;w\,).
\]

\bigskip

\textbf{Solution:}\\

Write the input vector in coordinates:
\begin{align}
\vec{v} = \myvec{x\\[4pt] y\\[4pt] z\\[4pt] w}
\end{align}

Each component of \(T(x,y,z,w)\) is a linear combination of \(x,y,z,w\). So , we can match the coefficients:

\begin{align}
T_1(x,y,z,w) &= 1\cdot x + 0\cdot y + 1\cdot z + 0\cdot w, \\[6pt]
T_2(x,y,z,w) &= 2\cdot x + 1\cdot y + 3\cdot z + 0\cdot w, \\[6pt]
T_3(x,y,z,w) &= 0\cdot x + 2\cdot y + 2\cdot z + 0\cdot w, \\[6pt]
T_4(x,y,z,w) &= 0\cdot x + 0\cdot y + 0\cdot z + 1\cdot w.
\end{align}

Therefore the matrix \(T\) is the matrix whose rows are the coefficient vectors:

\begin{align}
[T] \;=\;
	\myvec{
1 & 0 & 1 & 0 \\[4pt]
2 & 1 & 3 & 0 \\[4pt]
0 & 2 & 2 & 0 \\[4pt]
0 & 0 & 0 & 1
	}
\end{align}

On row reducing we get,
\begin{align}
\myvec{
1 & 0 & 1 & 0 \\[4pt]
0 & 1 & 1 & 0 \\[4pt]
0 & 0 & 0 & 1 \\[4pt]
0 & 0 & 0 & 0
}, \qquad
\operatorname{rank}([T]) = 3, \\[6pt]
\end{align}

\end{document}

