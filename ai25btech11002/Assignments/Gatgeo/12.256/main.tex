\let\negmedspace\undefined
\let\negthickspace\undefined
\documentclass[journal,12pt,onecolumn]{IEEEtran}
\usepackage{cite}
\usepackage{amsmath,amssymb,amsfonts,amsthm}
\usepackage{algorithmic}
\usepackage{graphicx}
\graphicspath{{Figs/}}
\usepackage{textcomp}
\usepackage{xcolor}
\usepackage{txfonts}
\usepackage{listings}
\usepackage{enumitem}
\usepackage{mathtools}
\usepackage{gensymb}
\usepackage{comment}
\usepackage{caption}
\usepackage[breaklinks=true]{hyperref}
\usepackage{tkz-euclide}
\usepackage{listings}
\usepackage{gvv}
\usepackage[latin1]{inputenc}
\usepackage{xparse}
\usepackage{color}
\usepackage{array}
\usepackage{longtable}
\usepackage{calc}
\usepackage{multirow}
\usepackage{multicol}
\usepackage{hhline}
\usepackage{ifthen}
\usepackage{lscape}
\usepackage{tabularx}
\usepackage{array}
\usepackage{float}

\providecommand{\myvec}[1]{\begin{pmatrix}#1\end{pmatrix}}

\begin{document}

\title{12.256}
\author{AI25BTECH11002 - Ayush Sunil Labhade}
{\let\newpage\relax\maketitle}

\textbf{Question:}\\
Time series \(\vec{P}\) and \(\vec{Q}\) are given by
\[
\vec{P} = \{1, -1, -2, 0, 1\}, \quad \vec{Q} = \{1, 0, -1\}.
\]
Find their linear convolution using the Toeplitz matrix method.

\bigskip

\textbf{Solution:}\\



\medskip
Represent the given sequences as:
\begin{align}
\vec{P} = \myvec{1 \\ -1 \\ -2 \\ 0 \\ 1}, \qquad
\vec{Q} = \myvec{1 & 0 & -1}
\end{align}

\medskip
The Toeplitz matrix corresponding to \(\vec{Q}\) is:
\begin{align}
\vec{R} =
\myvec{
1 & 0 & 0 & 0 & 0 \\
0 & 1 & 0 & 0 & 0 \\
-1 & 0 & 1 & 0 & 0 \\
0 & -1 & 0 & 1 & 0 \\
0 & 0 & -1 & 0 & 1 \\
0 & 0 & 0 & -1 & 0 \\
0 & 0 & 0 & 0 & -1
}
\end{align}

\medskip
Now, the convolution can be expressed as:
\begin{align}
\vec{Y} = \vec{R}\vec{P}
\end{align}

\medskip
Performing the matrix multiplication gives:
\begin{align}
\vec{Y} = \myvec{1 \\ -1 \\ -3 \\ 1 \\ 3 \\ 0 \\ -1}
\end{align}

\medskip
\[
\therefore\ \vec{P} * \vec{Q} = \{1, -1, -3, 1, 3, 0, -1\}
\]

\[
\text{Hence, the correct option is (b).}
\]

\end{document}

