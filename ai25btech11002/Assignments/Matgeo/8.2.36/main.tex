\let\negmedspace\undefined
\let\negthickspace\undefined
\documentclass[journal,12pt,onecolumn]{IEEEtran}
\usepackage{cite}
\usepackage{amsmath,amssymb,amsfonts,amsthm}
\usepackage{algorithmic}
\usepackage{graphicx}
\graphicspath{{Figs/}}
\usepackage{textcomp}
\usepackage{xcolor}
\usepackage{txfonts}
\usepackage{listings}
\usepackage{enumitem}
\usepackage{mathtools}
\usepackage{gensymb}
\usepackage{comment}
\usepackage{caption}
\usepackage[breaklinks=true]{hyperref}
\usepackage{tkz-euclide} 
\usepackage{listings}
\usepackage{gvv}                                        
%\def\inputGnumericTable{}                                 
\usepackage[latin1]{inputenc}     
\usepackage{xparse}
\usepackage{color}                                            
\usepackage{array}                                            
\usepackage{longtable}                                       
\usepackage{calc}                                             
\usepackage{multirow}
\usepackage{multicol}
\usepackage{hhline}                                           
\usepackage{ifthen}                                           
\usepackage{lscape}
\usepackage{tabularx}
\usepackage{array}
\usepackage{float}
%\newtheorem{theorem}{Theorem}[section]
%\newtheorem{theorem}{Theorem}[section]
%\newtheorem{problem}{Problem}
%\newtheorem{proposition}{Proposition}[section]
%\newtheorem{lemma}{Lemma}[section]
%\newtheorem{corollary}[theorem]{Corollary}
%\newtheorem{example}{Example}[section]
%\newtheorem{definition}[problem]{Definition}

\begin{document}


\title{8.2.36}
\author{AI25BTECH11002 - Ayush Sunil Labhade}
{\let\newpage\relax\maketitle}

\textbf{Question:} Find the equation of the conic that satisfies the given conditions.\\
Vertex $(0,4)$, Focus $(0,2)$.\\

\textbf{Solution:}\\
Since only one focus is given, the conic is a \textbf{parabola}.  
Let
\begin{align}
\vec{V_0} &= \myvec{0 \\ 4}, &
\vec{F} &= \myvec{0 \\ 2}.
\end{align}
As both lie on the Y-axis, the axis is vertical.  
Hence, let
\begin{align}
\vec{n} = \myvec{0 \\ 1}, \quad \norm{\vec{n}} = 1.
\end{align}

For the vertex point, by definition of a conic,
\begin{align}
\norm{\vec{V_0}-\vec{F}} &= e \frac{|\vec{n}^\top\vec{V_0} - c|}{\norm{\vec{n}}}.
\end{align}
Since $\norm{\vec{V_0}-\vec{F}} = 2$ and $e=1$,
\begin{align}
2 &= |4 - c| \Rightarrow c = 6.
\end{align}
Thus the directrix is $\vec{n}^T\vec{x} = 6$.\newline
(As $\vec{n}^T\vec{x}=2$ passes through the focus which is not possible)

The general matrix form of a conic is
\begin{align}
\vec{x}^\top\vec{V}\vec{x} + 2\vec{u}^\top\vec{x} + f = 0,
\end{align}
where
\begin{align}
\vec{V} &= \norm{\vec{n}}^2\vec{I} - e^2(\vec{n}\vec{n}^\top), &
\vec{u} &= ce^2\vec{n} - \norm{\vec{n}}^2\vec{F}, &
f &= \norm{\vec{n}}^2\norm{\vec{F}}^2 - c^2e^2.
\end{align}

Substituting $e=1$, $\vec{n}=\myvec{0\\1}$, $c=6$, $\vec{F}=\myvec{0\\2}$:
\begin{align}
\vec{V} &= \myvec{1 & 0 \\ 0 & 1} - \myvec{0 & 0 \\ 0 & 1} = \myvec{1 & 0 \\ 0 & 0},\\
\vec{u} &= 6\myvec{0 \\ 1} - \myvec{0 \\ 2} = \myvec{0 \\ 4},\\
f &= 4 - 36 = -32.
\end{align}

Hence,
\begin{align}
\vec{x}^\top\myvec{1 & 0 \\ 0 & 0}\vec{x} + 2\myvec{0 & 4}\vec{x} - 32 = 0,\\
x^2 + 8y - 32 = 0 \Rightarrow y =4- \frac{x^2}{8}.
\end{align}

Eigen decomposition of $\vec{V}$ gives
\begin{align}
\lambda_1 = 1, \ \lambda_2 = 0, \ 
\vec{p_1} = \myvec{1 \\ 0}, \ 
\vec{p_2} = \myvec{0 \\ 1}.
\end{align}
Since one eigenvalue is zero, the conic is confirmed as a \textbf{parabola}.
\newline
Graph:
\begin{figure}[H]
    \centering
    \includegraphics[scale=0.5]{plot}
    \caption{}
    \label{fig:plot}
\end{figure}
\end{document}
