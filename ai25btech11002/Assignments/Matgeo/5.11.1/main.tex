\let\negmedspace\undefined
\let\negthickspace\undefined
\documentclass[journal,12pt,onecolumn]{IEEEtran}
\usepackage{cite}
\usepackage{amsmath,amssymb,amsfonts,amsthm}
\usepackage{algorithmic}
\usepackage{graphicx}
\graphicspath{{Figs/}}
\usepackage{textcomp}
\usepackage{xcolor}
\usepackage{txfonts}
\usepackage{listings}
\usepackage{enumitem}
\usepackage{mathtools}
\usepackage{gensymb}
\usepackage{comment}
\usepackage{caption}
\usepackage[breaklinks=true]{hyperref}
\usepackage{tkz-euclide} 
\usepackage{listings}
\usepackage{gvv}                                        
%\def\inputGnumericTable{}                                 
\usepackage[latin1]{inputenc}     
\usepackage{xparse}
\usepackage{color}                                            
\usepackage{array}                                            
\usepackage{longtable}                                       
\usepackage{calc}                                             
\usepackage{multirow}
\usepackage{multicol}
\usepackage{hhline}                                           
\usepackage{ifthen}                                           
\usepackage{lscape}
\usepackage{tabularx}
\usepackage{array}
\usepackage{float}
%\newtheorem{theorem}{Theorem}[section]
%\newtheorem{theorem}{Theorem}[section]
%\newtheorem{problem}{Problem}
%\newtheorem{proposition}{Proposition}[section]
%\newtheorem{lemma}{Lemma}[section]
%\newtheorem{corollary}[theorem]{Corollary}
%\newtheorem{example}{Example}[section]
%\newtheorem{definition}[problem]{Definition}

\begin{document}


\title{4.7.63}
\author{AI25BTECH11002 - Ayush Sunil Labhade}
{\let\newpage\relax\maketitle}

\textbf{Question}:
Find the equation of plane that contains the point $\brak{1,-1,2}$ and is perpendicular to each of the planes 2x+3y-2z=5 and x+2y-3z=8. 	


\textbf{Solution:}

Let the equation of plane be:
\begin{align}
	\vec{n}^T\vec{x} = 1
\end{align}
Since the plane contains the point  $\vec{A} = \myvec{1\\-1\\2}$, 
\begin{align}
	\vec{n}^T\vec{A} = 1
\end{align}
Also since the plane is perpendicular to the planes 2x+3y-2z=5 and x+2y-3z=8
\begin{align}
	\vec{n}^T\vec{B} = 0 \newline \quad
\vec{B} = \myvec{2 \\ 3 \\ -2} \quad 
\end{align}
\begin{align}
	\vec{n}^T\vec{C} = 0 \newline \quad 
\vec{C} = \myvec{1 \\ 2 \\ -3}
\end{align}
We can rewrite it as
\begin{align}
	\brak{\vec{A} \vec{B} \vec{C}}^T\vec{n}=\myvec{1 \\ 0 \\0}
\end{align}
Forming the augmented matrix:
\begin{align}
	\augvec{3}{1}{1 & -1 & 2 & 1 \\ 2 & 3 & -2 & 0\\1 & 2 & -3& 0}
\end{align}
On row reducing we get,
\begin{align}
	\augvec{3}{1}{1 & 0 & 0 & -\tfrac{5}{7}\\
0 & 1 & 0 & \tfrac{4}{7}\\
0 & 0 & 1 & \tfrac{1}{7}}
\end{align}
	
\begin{align}
	\vec{n} = \myvec{-5 \\ 4 \\ 1}
\end{align}

$\therefore$ the required equation is
\begin{align}
		\myvec{-5&4&1}\vec{x} = 7
\end{align}
		

Graph:
\begin{figure}[H]
    \centering
    \includegraphics[scale=0.5]{plot}
    \caption{}
    \label{fig:plot}
\end{figure}
\end{document}
