\let\negmedspace\undefined
\let\negthickspace\undefined
\documentclass[journal,12pt,onecolumn]{IEEEtran}
\usepackage{cite}
\usepackage{amsmath,amssymb,amsfonts,amsthm}
\usepackage{algorithmic}
\usepackage{graphicx}
\graphicspath{{Figs/}}
\usepackage{textcomp}
\usepackage{xcolor}
\usepackage{txfonts}
\usepackage{listings}
\usepackage{enumitem}
\usepackage{mathtools}
\usepackage{gensymb}
\usepackage{comment}
\usepackage{caption}
\usepackage[breaklinks=true]{hyperref}
\usepackage{tkz-euclide} 
\usepackage{listings}
\usepackage{gvv}                                        
%\def\inputGnumericTable{}                                 
\usepackage[latin1]{inputenc}     
\usepackage{xparse}
\usepackage{color}                                            
\usepackage{array}                                            
\usepackage{longtable}                                       
\usepackage{calc}                                             
\usepackage{multirow}
\usepackage{multicol}
\usepackage{hhline}                                           
\usepackage{ifthen}                                           
\usepackage{lscape}
\usepackage{tabularx}
\usepackage{array}
\usepackage{float}
%\newtheorem{theorem}{Theorem}[section]
%\newtheorem{theorem}{Theorem}[section]
%\newtheorem{problem}{Problem}
%\newtheorem{proposition}{Proposition}[section]
%\newtheorem{lemma}{Lemma}[section]
%\newtheorem{corollary}[theorem]{Corollary}
%\newtheorem{example}{Example}[section]
%\newtheorem{definition}[problem]{Definition}

\begin{document}


\title{5.5.19}
\author{AI25BTECH11002 - Ayush Sunil Labhade}
{\let\newpage\relax\maketitle}


\textbf{Question}: 
Using elementary row transformations, find the inverse of the matrix
$$\myvec{2 & -3 & 5 \\ 3 & 2 & -4 \\ 1 & 1 & -2}$$

\textbf{Solution:}\\
Let $\vec{A} = \myvec{2 & -3 & 5 \\ 3 & 2 & -4 \\ 1 & 1 & -2}$\\

Augment the matrix $\vec{A}$ with the identity
\begin{align}
[\vec{A} \, | \, \vec{I}] =
\augvec{3}{3}{
2 & -3 & 5 & 1 & 0 & 0 \\
3 & 2 & -4 & 0 & 1 & 0 \\
1 & 1 & -2 & 0 & 0 & 1 \\
}
\end{align}

Row Transformation-1: $R_1 \rightarrow \frac{R_1}{2}$
\begin{align}
\augvec{3}{3}{
1 & -\frac{3}{2} & \frac{5}{2} & \frac{1}{2} & 0 & 0 \\
3 & 2 & -4 & 0 & 1 & 0 \\
1 & 1 & -2 & 0 & 0 & 1 \\
}
\end{align}

Row Transformation-2: $R_2 \rightarrow R_2 - 3R_1$
\begin{align}
\augvec{3}{3}{
1 & -\frac{3}{2} & \frac{5}{2} & \frac{1}{2} & 0 & 0 \\
0 & \frac{11}{2} & -\frac{23}{2} & -\frac{3}{2} & 1 & 0 \\
1 & 1 & -2 & 0 & 0 & 1 \\
}
\end{align}

Row Transformation-3: $R_3 \rightarrow R_3 - R_1$
\begin{align}
\augvec{3}{3}{
1 & -\frac{3}{2} & \frac{5}{2} & \frac{1}{2} & 0 & 0 \\
0 & \frac{11}{2} & -\frac{23}{2} & -\frac{3}{2} & 1 & 0 \\
0 & \frac{5}{2} & -\frac{9}{2} & -\frac{1}{2} & 0 & 1 \\
}
\end{align}

Row Transformation-4: $R_2 \rightarrow \frac{R_2}{\frac{11}{2}}$ (i.e., $R_2 \rightarrow \frac{2}{11}R_2$)
\begin{align}
\augvec{3}{3}{
1 & -\frac{3}{2} & \frac{5}{2} & \frac{1}{2} & 0 & 0 \\
0 & 1 & -\frac{23}{11} & -\frac{3}{11} & \frac{2}{11} & 0 \\
0 & \frac{5}{2} & -\frac{9}{2} & -\frac{1}{2} & 0 & 1 \\
}
\end{align}

Row Transformation-5: $R_3 \rightarrow R_3 - \frac{5}{2}R_2$
\begin{align}
\augvec{3}{3}{
1 & -\frac{3}{2} & \frac{5}{2} & \frac{1}{2} & 0 & 0 \\
0 & 1 & -\frac{23}{11} & -\frac{3}{11} & \frac{2}{11} & 0 \\
0 & 0 & \frac{11}{22} & \frac{1}{22} & -\frac{5}{11} & 1 \\
}
\end{align}

Row Transformation-6: $R_3 \rightarrow 22R_3$ (to clear the fraction)
\begin{align}
\augvec{3}{3}{
1 & -\frac{3}{2} & \frac{5}{2} & \frac{1}{2} & 0 & 0 \\
0 & 1 & -\frac{23}{11} & -\frac{3}{11} & \frac{2}{11} & 0 \\
0 & 0 & 1 & \frac{1}{11} & -\frac{10}{11} & 22 \\
}
\end{align}

Row Transformation-7: $R_2 \rightarrow R_2 + \frac{23}{11}R_3$
\begin{align}
\augvec{3}{3}{
1 & -\frac{3}{2} & \frac{5}{2} & \frac{1}{2} & 0 & 0 \\
0 & 1 & 0 & \frac{4}{11} & -\frac{33}{121} & \frac{22}{11} \\
0 & 0 & 1 & \frac{1}{11} & -\frac{10}{11} & 22 \\
}
\end{align}

Row Transformation-8: $R_1 \rightarrow R_1 + \frac{3}{2}R_2 - \frac{5}{2}R_3$
\begin{align}
\augvec{3}{3}{
1 & 0 & 0 & \frac{7}{11} & -\frac{35}{121} & \frac{33}{11} \\
0 & 1 & 0 & \frac{4}{11} & -\frac{33}{121} & \frac{22}{11} \\
0 & 0 & 1 & \frac{1}{11} & -\frac{10}{11} & 22 \\
}
\end{align}

The Inverse Matrix of $\vec{A}$:
\begin{align}
\vec{A}^{-1} = \myvec{ \dfrac{7}{11} & -\dfrac{35}{121} & \dfrac{33}{11} \\[1ex]
\dfrac{4}{11} & -\dfrac{33}{121} & \dfrac{22}{11} \\[1ex]
\dfrac{1}{11} & -\dfrac{10}{11} & 22 }
\end{align}

\end{document}  
