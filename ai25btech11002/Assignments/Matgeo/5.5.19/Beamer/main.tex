
\documentclass{beamer}
\mode<presentation>
\usepackage{amsmath}
\usepackage{amssymb}
%\usepackage{advdate}
\usepackage{graphicx}
\graphicspath{{../Figs/}}
\usepackage{adjustbox}
\usepackage{subcaption}
\usepackage{enumitem}
\usepackage{multicol}
\usepackage{mathtools}
\usepackage{listings}
\usepackage{url}
\def\UrlBreaks{\do\/\do-}
\usetheme{Boadilla}
\usecolortheme{lily}
\setbeamertemplate{footline}
{
  \leavevmode%
  \hbox{%
  \begin{beamercolorbox}[wd=\paperwidth,ht=2.25ex,dp=1ex,right]{author in head/foot}%
    \insertframenumber{} / \inserttotalframenumber\hspace*{2ex} 
  \end{beamercolorbox}}%
  \vskip0pt%
}
\setbeamertemplate{navigation symbols}{}
\let\solution\relax
\usepackage{gvv}
\lstset{
%language=C,
frame=single, 
breaklines=true,
columns=fullflexible
}

\numberwithin{equation}{section}

\begin{document}



\title{5.5.19}
\author{AI25BTECH11002 - Ayush Sunil Labhade}
{\let\newpage\relax\maketitle}


\textbf{Question}: 
Using elementary row transformations, find the inverse of the matrix
$$\myvec{2 & -3 & 5 \\ 3 & 2 & -4 \\ 1 & 1 & -2}$$

\textbf{Solution:}\\
Let $\vec{A} = \myvec{2 & -3 & 5 \\ 3 & 2 & -4 \\ 1 & 1 & -2}$\\

Augment the matrix $\vec{A}$ with the identity
\begin{align}
[\vec{A} \, | \, \vec{I}] =
\augvec{3}{3}{
2 & -3 & 5 & 1 & 0 & 0 \\
3 & 2 & -4 & 0 & 1 & 0 \\
1 & 1 & -2 & 0 & 0 & 1 \\
}
\end{align}

Row Transformation-1: $R_1 \rightarrow \frac{R_1}{2}$
\begin{align}
\augvec{3}{3}{
1 & -\frac{3}{2} & \frac{5}{2} & \frac{1}{2} & 0 & 0 \\
3 & 2 & -4 & 0 & 1 & 0 \\
1 & 1 & -2 & 0 & 0 & 1 \\
}
\end{align}

Row Transformation-2: $R_2 \rightarrow R_2 - 3R_1$
\begin{align}
\augvec{3}{3}{
1 & -\frac{3}{2} & \frac{5}{2} & \frac{1}{2} & 0 & 0 \\
0 & \frac{11}{2} & -\frac{23}{2} & -\frac{3}{2} & 1 & 0 \\
1 & 1 & -2 & 0 & 0 & 1 \\
}
\end{align}

Row Transformation-3: $R_3 \rightarrow R_3 - R_1$
\begin{align}
\augvec{3}{3}{
1 & -\frac{3}{2} & \frac{5}{2} & \frac{1}{2} & 0 & 0 \\
0 & \frac{11}{2} & -\frac{23}{2} & -\frac{3}{2} & 1 & 0 \\
0 & \frac{5}{2} & -\frac{9}{2} & -\frac{1}{2} & 0 & 1 \\
}
\end{align}

Row Transformation-4: $R_2 \rightarrow \frac{R_2}{\frac{11}{2}}$ (i.e., $R_2 \rightarrow \frac{2}{11}R_2$)
\begin{align}
\augvec{3}{3}{
1 & -\frac{3}{2} & \frac{5}{2} & \frac{1}{2} & 0 & 0 \\
0 & 1 & -\frac{23}{11} & -\frac{3}{11} & \frac{2}{11} & 0 \\
0 & \frac{5}{2} & -\frac{9}{2} & -\frac{1}{2} & 0 & 1 \\
}
\end{align}

Row Transformation-5: $R_3 \rightarrow R_3 - \frac{5}{2}R_2$
\begin{align}
\augvec{3}{3}{
1 & -\frac{3}{2} & \frac{5}{2} & \frac{1}{2} & 0 & 0 \\
0 & 1 & -\frac{23}{11} & -\frac{3}{11} & \frac{2}{11} & 0 \\
0 & 0 & \frac{11}{22} & \frac{1}{22} & -\frac{5}{11} & 1 \\
}
\end{align}

Row Transformation-6: $R_3 \rightarrow 22R_3$ (to clear the fraction)
\begin{align}
\augvec{3}{3}{
1 & -\frac{3}{2} & \frac{5}{2} & \frac{1}{2} & 0 & 0 \\
0 & 1 & -\frac{23}{11} & -\frac{3}{11} & \frac{2}{11} & 0 \\
0 & 0 & 1 & \frac{1}{11} & -\frac{10}{11} & 22 \\
}
\end{align}

Row Transformation-7: $R_2 \rightarrow R_2 + \frac{23}{11}R_3$
\begin{align}
\augvec{3}{3}{
1 & -\frac{3}{2} & \frac{5}{2} & \frac{1}{2} & 0 & 0 \\
0 & 1 & 0 & \frac{4}{11} & -\frac{33}{121} & \frac{22}{11} \\
0 & 0 & 1 & \frac{1}{11} & -\frac{10}{11} & 22 \\
}
\end{align}

Row Transformation-8: $R_1 \rightarrow R_1 + \frac{3}{2}R_2 - \frac{5}{2}R_3$
\begin{align}
\augvec{3}{3}{
1 & 0 & 0 & \frac{7}{11} & -\frac{35}{121} & \frac{33}{11} \\
0 & 1 & 0 & \frac{4}{11} & -\frac{33}{121} & \frac{22}{11} \\
0 & 0 & 1 & \frac{1}{11} & -\frac{10}{11} & 22 \\
}
\end{align}

The Inverse Matrix of $\vec{A}$:
\begin{align}
\vec{A}^{-1} = \myvec{ \dfrac{7}{11} & -\dfrac{35}{121} & \dfrac{33}{11} \\[1ex]
\dfrac{4}{11} & -\dfrac{33}{121} & \dfrac{22}{11} \\[1ex]
\dfrac{1}{11} & -\dfrac{10}{11} & 22 }
\end{align}
\end{document}
